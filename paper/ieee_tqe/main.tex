\documentclass[journal]{IEEEtran}

% Standard packages
\usepackage[T1]{fontenc}
\usepackage{amsmath}
\usepackage{amssymb}
\usepackage{graphicx}
\usepackage{booktabs}
\usepackage{hyperref}
\usepackage{cite}

% Hyperref setup (black links for IEEE)
\hypersetup{
    colorlinks=false,
    pdfborder={0 0 0}
}

% Custom commands
\newcommand{\ket}[1]{|#1\rangle}
\newcommand{\bra}[1]{\langle#1|}

\begin{document}

\title{Verified Implementation of GRAPE Pulse Optimization for Quantum Gates with Hardware-Representative Noise Models}

\author{\IEEEauthorblockN{Rylan Malarchick}
\IEEEauthorblockA{Department of Engineering Physics\\
Embry-Riddle Aeronautical University\\
Daytona Beach, FL 32114, USA\\
Email: malarchr@erau.edu}}

\maketitle

\begin{abstract}
Gate fidelity in noisy intermediate-scale quantum (NISQ) computers remains the primary bottleneck limiting practical quantum computation, constrained by decoherence and control noise. Quantum optimal control (QOC) techniques, particularly the gradient ascent pulse engineering (GRAPE) algorithm, offer a powerful approach to designing noise-robust pulses. However, most QOC implementations operate in idealized simulation environments that fail to capture hardware parameter drift, creating a critical ``sim-to-real'' gap. We present QubitPulseOpt, an open-source, rigorously-tested Python framework designed to bridge this gap through hardware-representative optimal control. The framework demonstrates API connectivity to IQM's Garnet quantum processor (20-qubit superconducting device) and implements a workflow constructing high-fidelity simulations using hardware-representative parameters. In closed-system simulation, GRAPE-optimized pulses achieve 99.14\% gate fidelity, representing a 77$\times$ error reduction compared to Gaussian baselines. The framework's reliability is ensured through an 864-test verification suite (74\% code coverage) adhering to NASA JPL Power-of-10 safety-critical coding standards.
\end{abstract}

\begin{IEEEkeywords}
Quantum optimal control, GRAPE algorithm, pulse optimization, superconducting qubits, Lindblad master equation, quantum gate fidelity
\end{IEEEkeywords}

\section{Introduction}
\IEEEPARstart{Q}{uantum} computing holds transformative potential across domains including cryptography, materials science, and drug discovery. However, the realization of fault-tolerant quantum computation remains hindered by fundamental error sources that corrupt quantum gate operations. The dominant error mechanisms in superconducting qubit architectures are energy relaxation (characterized by $T_1$) and dephasing (characterized by $T_2$), both stemming from unwanted coupling between the quantum system and its environment. These decoherence processes, combined with imperfections in classical control electronics and pulse distortions, severely limit gate fidelity. In the current NISQ era, typical two-qubit gate fidelities remain below 99\%, far from the $\sim$99.9\% threshold required for practical error correction.

Quantum optimal control (QOC) provides a principled framework for addressing this challenge by leveraging optimization algorithms to discover control pulse sequences that perform desired unitary operations while actively suppressing noise effects~\cite{khaneja2005optimal, glaser2015training}. The GRAPE (Gradient Ascent Pulse Engineering) algorithm~\cite{khaneja2005optimal} has emerged as a particularly powerful approach, using gradient-based optimization in pulse parameter space to maximize gate fidelity.

Despite these successes, a fundamental limitation persists: most QOC implementations optimize pulses using idealized or static device parameters. Real quantum hardware exhibits temporal drift in critical parameters such as qubit frequencies, anharmonicities, and coherence times. This ``sim-to-real gap''---the mismatch between simulation assumptions and physical reality---represents a critical barrier to translating QOC from theoretical promise to practical utility.

In this paper, we present QubitPulseOpt, an open-source framework for designing noise-robust quantum control pulses that directly addresses the sim-to-real gap. The framework implements a workflow that: (1) demonstrates API connectivity to quantum cloud platforms, (2) constructs Lindblad master equation simulations incorporating hardware-representative noise parameters, and (3) executes GRAPE optimization to discover custom pulse sequences. QubitPulseOpt distinguishes itself through software-engineering rigor: 864 tests achieving 74\% code coverage, with adherence to NASA JPL Power-of-10 safety-critical coding standards.

\section{Theoretical and Computational Model}

\subsection{System Hamiltonian}

The quantum system under consideration is a superconducting transmon qubit. The time-dependent Hamiltonian governing system dynamics decomposes into drift and control components:
\begin{equation}
H(t) = H_d + H_c(t),
\end{equation}
where $H_d$ represents the intrinsic system Hamiltonian and $H_c(t)$ encodes time-dependent microwave control fields. In the rotating frame:
\begin{equation}
H_d = \omega_q a^\dagger a + \frac{\alpha}{2} a^\dagger a (a^\dagger a - 1),
\end{equation}
where $\omega_q$ is the qubit transition frequency, $\alpha < 0$ is the anharmonicity (typically $|\alpha|/2\pi \approx 200$--$300$ MHz), and $a$ ($a^\dagger$) are the annihilation (creation) operators.

The control Hamiltonian describes interaction with applied microwave pulses:
\begin{equation}
H_c(t) = \Omega_I(t) (a + a^\dagger) + \Omega_Q(t) (ia^\dagger - ia),
\end{equation}
where $\Omega_I(t)$ and $\Omega_Q(t)$ are time-dependent in-phase and quadrature control amplitudes.

\subsection{Open System Dynamics}

System evolution is described by the Lindblad master equation~\cite{lindblad1976generators}:
\begin{equation}
\frac{d\rho}{dt} = -i[H(t), \rho] + \sum_k \gamma_k \left( L_k \rho L_k^\dagger - \frac{1}{2} \{L_k^\dagger L_k, \rho\} \right),
\label{eq:lindblad}
\end{equation}
where $\rho$ is the density matrix and $L_k$ are jump operators with rates $\gamma_k$. For the transmon, we include energy relaxation ($L_1 = \sqrt{\gamma_1} a$, $\gamma_1 = 1/T_1$) and pure dephasing ($L_2 = \sqrt{\gamma_\phi} a^\dagger a$, $\gamma_\phi = 1/T_2 - 1/(2T_1)$).

\subsection{The GRAPE Algorithm}

GRAPE~\cite{khaneja2005optimal} is a gradient-based optimization method for discovering control pulses that maximize gate fidelity:
\begin{equation}
F = \frac{1}{d} \left| \text{Tr}(U_{\text{target}}^\dagger U_{\text{final}}) \right|^2,
\end{equation}
where $U_{\text{target}}$ is the desired unitary gate, $U_{\text{final}}$ is the propagator from the optimized pulse, and $d$ is the Hilbert space dimension.

The total gate duration $T$ is discretized into $N$ time slices. The control amplitudes in each slice become optimization variables. The algorithm computes fidelity gradients using efficient forward-backward propagation~\cite{khaneja2005optimal}:
\begin{equation}
\Omega^{(k+1)} = \Omega^{(k)} + \epsilon \nabla_{\Omega^{(k)}} F.
\end{equation}
We employ L-BFGS-B, a quasi-Newton method well-suited to moderate-dimensional optimization with bound constraints.

\section{The QubitPulseOpt Framework}

\subsection{Hardware-Calibrated Workflow}

The central innovation is an automated, hardware-calibrated workflow:

\begin{enumerate}
\item \textbf{Parameter Query:} The system queries cloud APIs of target QPUs (IQM Garnet, 20-qubit superconducting system) to retrieve $T_1$, $T_2$, $\omega_q$, and $\alpha$.

\item \textbf{Digital Twin Instantiation:} Retrieved parameters are injected into the Lindblad solver, constructing a device-specific simulation.

\item \textbf{GRAPE Optimization:} The algorithm discovers pulse sequences tailored to specific device characteristics.

\item \textbf{Validation:} Optimized pulses are compared against standard shapes in identical hardware-calibrated environments.
\end{enumerate}

\subsection{Software Verification and Validation}

QubitPulseOpt incorporates comprehensive V\&V infrastructure:

\begin{itemize}
\item \textbf{Unit Test Suite:} 864 tests covering all major functionality, achieving 74\% overall code coverage with over 85\% coverage of critical hardware integration modules.

\item \textbf{Continuous Integration:} Automated testing on every commit.

\item \textbf{Power-of-10 Compliance:} Adherence to NASA JPL's safety-critical coding standards~\cite{holzmann2006power}.

\item \textbf{Benchmark Validation:} Validation against analytical results (Rabi oscillations, free decay).
\end{itemize}

\section{Results and Validation}

\subsection{Pulse Optimization}

GRAPE optimization was executed for a target X-gate with gate time $T = 20$ ns, discretized into $N = 100$ time steps. Optimization used L-BFGS-B with amplitude constraints $|\Omega_{I,Q}| \leq 2\pi \times 50$ MHz.

Fig.~\ref{fig:optimization} shows fidelity convergence and pulse comparison. The algorithm converges from random initial pulse to $F > 0.99$ within 200 iterations. The optimized pulse exhibits highly non-trivial temporal structure with rapid amplitude modulation---features emerging naturally from gradient-based search.

\begin{figure}[t]
\centering
\includegraphics[width=0.9\columnwidth]{figures/verified_fidelity_convergence.png}
\caption{GRAPE optimization convergence over 200 iterations, reaching 99.14\% fidelity from random initial pulse.}
\label{fig:optimization}
\end{figure}

\subsection{Hardware-Representative Validation}

Simulation used hardware-representative parameters typical of IQM Garnet:
\begin{itemize}
\item $T_1 = 50~\mu$s, $T_2 = 70~\mu$s
\item $\omega_q/2\pi = 5.0$ GHz, $\alpha/2\pi = -300$ MHz
\end{itemize}

Results demonstrate dramatic performance difference:
\begin{itemize}
\item \textbf{Gaussian pulse:} $F = 0.334$ (66.60\% error)
\item \textbf{GRAPE-optimized:} $F = 0.9914$ (0.86\% error)
\item \textbf{Error reduction:} 77$\times$
\end{itemize}

\begin{figure}[t]
\centering
\includegraphics[width=0.9\columnwidth]{figures/verified_error_comparison.png}
\caption{Gate error comparison. Gaussian baseline: 66.60\% error. GRAPE-optimized: 0.86\% error, demonstrating 77$\times$ reduction.}
\label{fig:error}
\end{figure}

\subsection{Limitations}

\textbf{Closed-System Approximation:} GRAPE optimizations were performed in closed-system approximation. The 99.14\% fidelity represents idealized performance; hardware fidelity under decoherence will be lower.

\textbf{Baseline Context:} The 77$\times$ improvement reflects comparison against an uncalibrated Gaussian baseline. Literature-typical GRAPE improvements over properly calibrated DRAG pulses are 2--10$\times$.

\textbf{No Hardware Execution:} All results are simulation-only. API connectivity to IQM Garnet was confirmed, but pulse execution on physical hardware was not performed.

\section{Conclusion}

We present QubitPulseOpt, an open-source framework addressing a critical gap in quantum optimal control: reliable, verified software for pulse optimization with hardware-representative noise models. GRAPE-optimized pulses achieve 77$\times$ error reduction in simulation compared to standard pulses.

Beyond algorithmic contributions, QubitPulseOpt establishes a new standard for quantum software engineering through its 864-test verification suite and adherence to safety-critical coding standards. Future work will implement open-system GRAPE, extend to two-qubit gates, and validate on physical hardware.

\section*{Acknowledgment}

The author thanks Embry-Riddle Aeronautical University for research support. QubitPulseOpt is available at \url{https://github.com/rylanmalarchick/QubitPulseOpt}.

\begin{thebibliography}{10}

\bibitem{khaneja2005optimal}
N. Khaneja, T. Reiss, C. Kehlet, T. Schulte-Herbr\"uggen, and S. J. Glaser,
``Optimal control of coupled spin dynamics: design of NMR pulse sequences by gradient ascent algorithms,''
\textit{J. Magn. Reson.}, vol. 172, no. 2, pp. 296--305, 2005.

\bibitem{glaser2015training}
S. J. Glaser \textit{et al.},
``Training Schr\"odinger's cat: quantum optimal control,''
\textit{Eur. Phys. J. D}, vol. 69, no. 12, p. 279, 2015.

\bibitem{lindblad1976generators}
G. Lindblad,
``On the generators of quantum dynamical semigroups,''
\textit{Commun. Math. Phys.}, vol. 48, no. 2, pp. 119--130, 1976.

\bibitem{holzmann2006power}
G. J. Holzmann,
``The power of 10: Rules for developing safety-critical code,''
\textit{Computer}, vol. 39, no. 6, pp. 95--99, 2006.

\end{thebibliography}

\end{document}
