\documentclass[11pt]{letter}
\usepackage[margin=1in]{geometry}
\usepackage{hyperref}

\signature{Rylan Malarchick\\Department of Engineering Physics\\Embry-Riddle Aeronautical University}

\address{Rylan Malarchick\\Department of Engineering Physics\\Embry-Riddle Aeronautical University\\Daytona Beach, FL 32114, USA\\malarchr@erau.edu}

\begin{document}

\begin{letter}{Editorial Office\\Physical Review A\\American Physical Society}

\opening{Dear Editors,}

I submit the enclosed manuscript, ``Numerical GRAPE optimization of single-qubit gates in a three-level transmon model: Leakage suppression and error budget analysis with hardware-representative parameters,'' for consideration as a Regular Article in Physical Review A.

This manuscript presents a systematic numerical comparison of Gaussian, DRAG, and GRAPE pulse optimization methods for single-qubit gates in a three-level transmon model. Using hardware-representative parameters from IQM's Garnet processor ($T_1 = 37\;\mu$s, $T_2 = 9.6\;\mu$s, $\alpha/2\pi = -200$ MHz), the paper provides four results:

\begin{enumerate}
\item \textbf{Complete leakage suppression by GRAPE}: In the closed three-level system at 20 ns gate time, GRAPE achieves unit fidelity ($1 - F < 10^{-15}$) with negligible leakage ($P_2 < 10^{-16}$), while Gaussian pulses retain $F = 0.972$ and properly calibrated DRAG achieves $F = 0.9995$.

\item \textbf{Decoherence floor identification}: An error budget analysis using the Lindblad master equation decomposes gate error into coherent, $T_1$, $T_2$, and control noise contributions. GRAPE reaches the decoherence floor ($\epsilon = 7.2 \times 10^{-4}$), outperforming Gaussian by $39\times$. Properly calibrated DRAG achieves $\epsilon = 8.4 \times 10^{-4}$, only $1.2\times$ above GRAPE, showing that DRAG is already near the decoherence floor when correctly implemented.

\item \textbf{Correct DRAG calibration matters}: The analytical DRAG parameter $\beta = -1/(2\alpha)$ yields an amplitude-independent, $O(1)$ quantity ($\beta = 0.398$) that provides effective first-order leakage suppression improving monotonically with gate time, achieving $F > 0.999999$ at 100 ns.

\item \textbf{Robustness tradeoffs}: DRAG exhibits superior detuning robustness compared to both Gaussian and GRAPE (minimum fidelity 0.990 vs.\ 0.931 for GRAPE over $\pm 5$ MHz), while GRAPE retains the best amplitude robustness (minimum 0.994). This has practical implications for choosing control strategies based on the dominant calibration uncertainty.
\end{enumerate}

This work addresses a gap in the literature: while both DRAG and GRAPE are widely used, side-by-side comparisons under identical multi-level conditions with hardware-representative parameters remain sparse. The results show that properly calibrated DRAG and GRAPE have complementary strengths, rather than numerical optimization uniformly outperforming analytical methods.

All code and data are available as the open-source QubitPulseOpt framework at \url{https://github.com/rylanmalarchick/QubitPulseOpt}. A preprint of an earlier version is available at arXiv:2511.12799; this submission is substantially revised with corrected DRAG implementation, new experiments, and updated analysis.

This manuscript has not been submitted elsewhere.

Thank you for considering this manuscript.

\closing{Sincerely,}

\end{letter}
\end{document}
