\documentclass[11pt]{letter}
\usepackage[margin=1in]{geometry}
\usepackage{hyperref}

\signature{Rylan Malarchick\\Department of Engineering Physics\\Embry-Riddle Aeronautical University}

\address{Rylan Malarchick\\Department of Engineering Physics\\Embry-Riddle Aeronautical University\\Daytona Beach, FL 32114, USA\\malarchr@erau.edu}

\begin{document}

\begin{letter}{Editorial Office\\Physical Review A\\American Physical Society}

\opening{Dear Editors,}

We submit the enclosed manuscript, ``Numerical GRAPE optimization of single-qubit gates in a three-level transmon model: Leakage suppression and error budget analysis with hardware-representative parameters,'' for consideration as a Regular Article in Physical Review A.

This manuscript presents a systematic numerical comparison of Gaussian, DRAG, and GRAPE pulse optimization methods for single-qubit gates in a three-level transmon model---the minimal model that captures leakage physics in superconducting qubits. Using hardware-representative parameters from IQM's Garnet processor, we provide four key results:

\begin{enumerate}
\item \textbf{Complete leakage suppression by GRAPE}: In the three-level transmon, GRAPE achieves unit closed-system fidelity ($1 - F < 10^{-15}$) with negligible leakage to the $|2\rangle$ state across gate times from 10 to 100 ns, while Gaussian and DRAG pulses retain significant coherent errors (2.8\% and 12.8\% infidelity, respectively, at 20 ns).

\item \textbf{Decoherence floor identification}: An error budget analysis using the Lindblad master equation decomposes gate error into coherent, $T_1$, $T_2$, and control noise contributions. GRAPE reaches the decoherence floor ($\epsilon = 7.2 \times 10^{-4}$), outperforming Gaussian by 39$\times$ and DRAG by 178$\times$ in the three-level model.

\item \textbf{DRAG overcorrection anomaly}: We show that the analytical DRAG parameter $\beta = -\alpha/(2\Omega_\text{max})$ leads to monotonically \emph{decreasing} fidelity at longer gate times due to overcorrection---a result highlighting the limited validity regime of first-order perturbative corrections.

\item \textbf{Robustness characterization}: GRAPE pulses exhibit excellent amplitude robustness ($F > 0.994$ over $\pm 5\%$ error) but modest detuning sensitivity, motivating future robust optimization approaches.
\end{enumerate}

We believe this work makes a timely contribution to Physical Review A because it addresses a gap in the literature: while both DRAG and GRAPE are widely used, rigorous side-by-side comparisons under identical multi-level conditions with realistic hardware parameters are surprisingly sparse. Our error budget analysis provides a clear picture of which error sources limit gate performance in modern transmon hardware and where numerical optimization offers genuine advantages over analytical corrections.

The work is novel in its systematic approach and in identifying the DRAG overcorrection anomaly. All code and data are open source. A preprint of an earlier version is available at arXiv:2511.12799; this submission is substantially revised with new experiments and analysis.

This manuscript has not been submitted elsewhere. All authors have approved the manuscript for submission.

We suggest the following referees who are experts in quantum optimal control for superconducting qubits:
\begin{itemize}
\item Frank K. Wilhelm (Forschungszentrum J\"ulich) -- pioneer of DRAG and optimal control theory
\item Felix Motzoi (Forschungszentrum J\"ulich) -- co-inventor of DRAG
\item Christiane P. Koch (Freie Universit\"at Berlin) -- quantum optimal control theory
\item David I. Schuster (Stanford University) -- experimental quantum control
\end{itemize}

Thank you for considering our manuscript.

\closing{Sincerely,}

\end{letter}
\end{document}
