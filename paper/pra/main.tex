\documentclass[twocolumn,showpacs,preprintnumbers,amsmath,amssymb,pra,floatfix]{revtex4-2}

% Standard packages
\usepackage[T1]{fontenc}
\usepackage{graphicx}
\usepackage{booktabs}
\usepackage{hyperref}
\usepackage{siunitx}
\usepackage{algorithm}
\usepackage{algpseudocode}
\usepackage{bm}
\usepackage{dcolumn}

\hypersetup{
    colorlinks=true,
    linkcolor=blue,
    citecolor=blue,
    urlcolor=blue
}

\begin{document}

\preprint{arXiv:2511.12799v2}

\title{Numerical GRAPE optimization of single-qubit gates in a three-level transmon model:\\Leakage suppression and error budget analysis with hardware-representative parameters}

\author{Rylan Malarchick}
\affiliation{Department of Engineering Physics, Embry-Riddle Aeronautical University, Daytona Beach, FL 32114, USA}
\email{malarchr@erau.edu}

\date{\today}

\begin{abstract}
We present a systematic numerical comparison of Gaussian, derivative removal by adiabatic gate (DRAG), and gradient ascent pulse engineering (GRAPE) control pulses for single-qubit gates on a three-level transmon model with hardware-representative parameters from IQM's Garnet processor ($T_1 = \SI{37}{\micro\second}$, $T_2 = \SI{9.6}{\micro\second}$, $\alpha/2\pi = \SI{-200}{\mega\hertz}$). In the closed three-level system at a gate time of \SI{20}{\nano\second}, GRAPE achieves unit fidelity ($1 - F < 10^{-15}$) with negligible leakage to the $\ket{2}$ state ($P_2 < 10^{-16}$), compared to $F = 0.972$ for Gaussian pulses ($P_2 = 6.4 \times 10^{-4}$) and $F = 0.872$ for DRAG ($P_2 = 1.0 \times 10^{-4}$). An error budget analysis using the Lindblad master equation shows that GRAPE's coherent error vanishes, placing it at the decoherence floor ($1 - F = 7.2 \times 10^{-4}$), while Gaussian and DRAG pulses are dominated by coherent errors $39\times$ and $178\times$ larger, respectively. A gate-time sweep from 10 to \SI{100}{\nano\second} reveals that GRAPE maintains unit closed-system fidelity across all gate times, while Gaussian fidelity improves monotonically and DRAG fidelity \emph{decreases} at longer gate times due to overcorrection from the analytical $\beta$ parameter. Robustness analysis shows GRAPE retains superior amplitude-error tolerance in the three-level system (minimum fidelity 0.994 over $\pm 5\%$ amplitude variation) but exhibits modestly increased sensitivity to detuning compared to Gaussian pulses, motivating future robust optimization approaches. All simulations and the open-source QubitPulseOpt framework are available at \url{https://github.com/rylanmalarchick/QubitPulseOpt}.
\end{abstract}

\pacs{03.67.Lx, 85.25.Cp, 03.65.Yz}

\maketitle

%==============================================================================
\section{Introduction}
\label{sec:introduction}
%==============================================================================

High-fidelity single-qubit gates are a prerequisite for fault-tolerant quantum computation~\cite{preskill2018quantum, knill2005quantum}. In superconducting transmon qubits~\cite{koch2007charge}, the weakly anharmonic energy spectrum creates a persistent tension: fast gates require strong driving fields that couple the computational subspace $\{\ket{0}, \ket{1}\}$ to higher transmon levels, introducing leakage errors that are not correctable by standard quantum error correction codes~\cite{aliferis2007fault, ghosh2013understanding}. This leakage-speed tradeoff is the central challenge for single-qubit gate optimization in transmon architectures.

The derivative removal by adiabatic gate (DRAG) protocol~\cite{motzoi2009simple, gambetta2011analytic} provides a first-order analytical correction that adds a derivative component to the quadrature control channel, suppressing transitions to the $\ket{2}$ state. DRAG has been widely adopted in experimental practice and achieves high fidelities when properly calibrated~\cite{chen2016measuring, sheldon2016procedure}. However, DRAG is a perturbative correction valid to first order in $\Omega/\alpha$ (the ratio of Rabi frequency to anharmonicity) and does not account for higher-order leakage pathways or non-perturbative effects at short gate times.

Numerical optimal control methods, particularly the gradient ascent pulse engineering (GRAPE) algorithm~\cite{khaneja2005optimal}, can in principle discover pulse shapes that suppress leakage to arbitrary order by optimizing directly over the full multi-level Hilbert space. GRAPE has been successfully applied to superconducting qubit gates~\cite{lucero2010reduced, kelly2014optimal, werninghaus2021leakage, propson2022robust}, nitrogen-vacancy centers~\cite{dolde2014high}, and trapped ions~\cite{nebendahl2009optimal}. The key advantage of numerical optimization is its ability to exploit the full control landscape without perturbative approximations.

Despite the extensive literature on both DRAG and GRAPE, systematic comparisons under identical conditions with hardware-representative parameters remain surprisingly sparse. Many GRAPE demonstrations compare against uncalibrated baselines, while DRAG studies focus on two-level models where leakage is absent by construction. A rigorous comparison requires: (i) a multi-level transmon model that captures leakage physics, (ii) realistic device parameters from contemporary hardware, (iii) an error budget decomposing coherent and incoherent contributions, and (iv) robustness analysis against calibration imperfections.

In this paper, we address this gap through four systematic experiments using a three-level transmon model parameterized by calibration data representative of IQM's Garnet processor~\cite{iqm2024garnet}. We compare Gaussian, DRAG, and GRAPE pulses for the X gate across gate times from 10 to \SI{100}{\nano\second}, decompose the error budget into coherent, $T_1$, $T_2$, and control noise contributions using the Lindblad master equation~\cite{lindblad1976generators, breuer2002theory}, and analyze robustness to detuning and amplitude errors. Our results quantify the regime where GRAPE's advantage over analytical corrections is most significant and identify the decoherence floor that bounds achievable fidelity regardless of pulse optimization.

The remainder of this paper is organized as follows. Section~\ref{sec:theory} presents the theoretical model. Section~\ref{sec:methods} describes the computational methods and pulse construction. Section~\ref{sec:results} presents results from all four experiments. Section~\ref{sec:discussion} discusses the findings in context. Section~\ref{sec:conclusion} concludes with future directions.

%==============================================================================
\section{Theoretical Model}
\label{sec:theory}
%==============================================================================

\subsection{Three-level transmon Hamiltonian}

We model the transmon as a weakly anharmonic oscillator truncated to three levels ($\ket{0}$, $\ket{1}$, $\ket{2}$). In the frame rotating at the qubit frequency $\omega_q$, the drift Hamiltonian is~\cite{koch2007charge}
\begin{equation}
H_d = \frac{\alpha}{2} \hat{n}(\hat{n} - \hat{I}),
\label{eq:drift}
\end{equation}
where $\alpha < 0$ is the anharmonicity and $\hat{n} = a^\dagger a$ is the number operator, with $a$ ($a^\dagger$) the bosonic annihilation (creation) operators truncated to the three-level subspace. In matrix form, $H_d = \text{diag}(0, 0, \alpha)$, so the $\ket{0} \leftrightarrow \ket{1}$ transition is resonant in the rotating frame and the $\ket{1} \leftrightarrow \ket{2}$ transition is detuned by $\alpha$.

The control Hamiltonian describes microwave driving through in-phase ($I$) and quadrature ($Q$) channels:
\begin{equation}
H_c(t) = \Omega_I(t) H_x + \Omega_Q(t) H_y,
\label{eq:control}
\end{equation}
where $\Omega_I(t)$ and $\Omega_Q(t)$ are the time-dependent control amplitudes (in rad/ns) and
\begin{align}
H_x &= \tfrac{1}{2}(a + a^\dagger), \label{eq:hx} \\
H_y &= \tfrac{1}{2}i(a^\dagger - a). \label{eq:hy}
\end{align}
The factor of $1/2$ follows the convention where $\Omega_I$ is the Rabi frequency. The total Hamiltonian is $H(t) = H_d + H_c(t)$.

For the two-level model used in validation experiments, we set $H_d = 0$ (resonant driving in the rotating frame) and replace $H_x, H_y$ with $\sigma_x/2, \sigma_y/2$.

\subsection{Pulse parameterization}

\subsubsection{Gaussian pulse}

The Gaussian pulse drives only the $I$-channel:
\begin{equation}
\Omega_I(t) = A \exp\left[-\frac{(t - T/2)^2}{2\sigma^2}\right], \quad \Omega_Q(t) = 0,
\label{eq:gaussian}
\end{equation}
where $T$ is the gate duration, $\sigma = T/(2n_\sigma)$ with $n_\sigma = 4$, and the amplitude $A$ is chosen to achieve a $\pi$-rotation:
\begin{equation}
A = \frac{\pi}{\sigma\sqrt{2\pi}}.
\label{eq:gaussian_amp}
\end{equation}

\subsubsection{DRAG pulse}

The DRAG protocol~\cite{motzoi2009simple} adds a derivative correction on the quadrature channel:
\begin{equation}
\Omega_Q(t) = \beta \frac{d\Omega_I}{dt},
\label{eq:drag}
\end{equation}
where the DRAG parameter is
\begin{equation}
\beta = -\frac{\alpha}{2\Omega_\text{max}}
\label{eq:beta}
\end{equation}
with $\alpha$ the anharmonicity and $\Omega_\text{max}$ the peak Rabi frequency, both in the same angular frequency units. For our parameters ($\alpha/2\pi = -200$ MHz, $\Omega_\text{max} = 0.251$ rad/ns), this gives $\beta \approx 2.5$.

Note that Eq.~\eqref{eq:beta} requires careful unit consistency: if $\alpha$ is specified in MHz, it must be converted to rad/ns via $\alpha_\text{rad/ns} = 2\pi \times \alpha_\text{MHz} \times 10^{-3}$ before computing $\beta$.

\subsubsection{GRAPE pulse}

The GRAPE algorithm~\cite{khaneja2005optimal} represents the control pulse as piecewise-constant amplitudes $\{\Omega_I^{(k)}, \Omega_Q^{(k)}\}$ for $k = 1, \ldots, N$ time slices of duration $\delta t = T/N$. These $2N$ parameters are optimized to maximize the gate fidelity (Sec.~\ref{sec:fidelity}).

\subsection{Gate fidelity}
\label{sec:fidelity}

For closed-system (unitary) evolution, we define the gate fidelity as
\begin{equation}
F = \frac{1}{d^2} \left|\text{Tr}(U_\text{target}^\dagger U_\text{final})\right|^2,
\label{eq:fidelity}
\end{equation}
where $U_\text{final}$ is the propagator resulting from the applied pulse, $U_\text{target}$ is the desired unitary, and $d$ is the dimension of the target subspace. For three-level simulations targeting a two-qubit gate (e.g., the X gate acting on $\{\ket{0}, \ket{1}\}$), $U_\text{target}$ is the $2 \times 2$ gate embedded as the upper-left block of a $3 \times 3$ unitary with identity action on $\ket{2}$, and $d = 2$. The fidelity is evaluated by projecting $U_\text{final}$ onto the computational subspace.

Leakage probability is defined as
\begin{equation}
P_2 = 1 - \sum_{j \in \{0,1\}} |\langle j | U_\text{final} | \psi_\text{init} \rangle|^2,
\label{eq:leakage}
\end{equation}
averaged over initial states $\ket{0}$ and $\ket{1}$.

\subsection{Open-system dynamics}
\label{sec:lindblad}

Decoherence is modeled by the Lindblad master equation~\cite{lindblad1976generators}:
\begin{equation}
\frac{d\rho}{dt} = -i[H(t), \rho] + \sum_k \left(L_k \rho L_k^\dagger - \frac{1}{2}\{L_k^\dagger L_k, \rho\}\right),
\label{eq:lindblad}
\end{equation}
with collapse operators for amplitude damping ($T_1$) and pure dephasing:
\begin{align}
L_1 &= \sqrt{\gamma_1}\, a, \quad \gamma_1 = 1/T_1, \label{eq:t1} \\
L_2 &= \sqrt{\gamma_\phi}\, \hat{n}, \quad \gamma_\phi = 1/T_2 - 1/(2T_1). \label{eq:t2}
\end{align}
For the three-level system, $a$ is the $3 \times 3$ annihilation operator and $\hat{n}$ is the number operator, naturally generalizing the two-level $\sigma_-$ and $\sigma_z/2$ collapse operators. The constraint $T_2 \leq 2T_1$ ensures $\gamma_\phi \geq 0$.

The process fidelity under open-system evolution is computed as
\begin{equation}
F_\text{proc} = \frac{1}{d} \sum_{j=0}^{d-1} \langle j | U_\text{target}^\dagger \rho_j(T) U_\text{target} | j \rangle,
\label{eq:process_fidelity}
\end{equation}
where $\rho_j(T)$ is the final density matrix when initializing in $\ket{j}$.

%==============================================================================
\section{Computational Methods}
\label{sec:methods}
%==============================================================================

\subsection{GRAPE optimization}

We implement the GRAPE algorithm following Khaneja \textit{et al.}~\cite{khaneja2005optimal}. The total propagator is constructed as
\begin{equation}
U_\text{final} = \prod_{k=N}^{1} U_k, \quad U_k = \exp\left(-i H_k \delta t\right),
\label{eq:propagator}
\end{equation}
where $H_k = H_d + \Omega_I^{(k)} H_x + \Omega_Q^{(k)} H_y$ is the Hamiltonian during the $k$-th time slice. The gradient of the fidelity with respect to the control amplitude $\Omega_c^{(k)}$ (for control channel $c \in \{I, Q\}$) is computed as~\cite{khaneja2005optimal, degroot2024grape}
\begin{equation}
\frac{\partial F}{\partial \Omega_c^{(k)}} = \frac{2}{d^2} \text{Re}\left[\text{Tr}(U_\text{target}^\dagger P_k)\, \text{Tr}(X_k^\dagger U_\text{target})\right],
\label{eq:gradient}
\end{equation}
where $P_k = U_N \cdots U_{k+1}$ is the forward propagator from slice $k+1$ to the end, and $X_k = -i \delta t\, H_c\, U_k\, U_{k-1} \cdots U_1$ incorporates the derivative of the $k$-th propagator. This first-order approximation $\partial U_k / \partial \Omega_c^{(k)} \approx -i \delta t\, H_c\, U_k$ is standard for GRAPE and is accurate when $\|H_k\| \delta t \ll 1$.

Optimization uses gradient ascent with momentum:
\begin{equation}
\Omega^{(k)}_{n+1} = \Omega^{(k)}_n + \eta \nabla F + \mu \Delta \Omega^{(k)}_{n-1},
\end{equation}
where $\eta$ is the learning rate and $\mu$ is the momentum coefficient. For the two-level system, we use $\eta = 0.1$, $\mu = 0.9$, and a step decay factor of 0.99 per iteration. For the three-level system, we use $\eta = 0.5$, $\mu = 0.5$, and no step decay (factor 1.0), as the larger Hilbert space and leakage landscape require more aggressive optimization.

The pulse is discretized into $N = 50$ time slices for a \SI{20}{\nano\second} gate ($\delta t = \SI{0.4}{\nano\second}$). Initialization uses a Gaussian seed on the $I$-channel for X and Y gates, with gate-specific seeds for other targets (Hadamard, S, T).

\subsection{Lindblad integration}

Open-system dynamics are solved using QuTiP's \texttt{mesolve}~\cite{johansson2012qutip, johansson2013qutip} with the Lindblad master equation [Eq.~\eqref{eq:lindblad}]. For the three-level system, collapse operators are constructed as described in Eqs.~\eqref{eq:t1}--\eqref{eq:t2} using the appropriate bosonic operators.

\subsection{Hardware-representative parameters}

All simulations use parameters representative of IQM's Garnet 20-qubit superconducting processor~\cite{iqm2024garnet}:

\begin{table}[!htbp]
\caption{Simulation parameters representative of IQM Garnet.}
\label{tab:params}
\begin{ruledtabular}
\begin{tabular}{lc}
Parameter & Value \\
\hline
Qubit frequency $\omega_q/2\pi$ & \SI{5.0}{\giga\hertz} \\
Anharmonicity $\alpha/2\pi$ & \SI{-200}{\mega\hertz} \\
$T_1$ & \SI{37}{\micro\second} \\
$T_2$ & \SI{9.6}{\micro\second} \\
Gate duration $T$ & \SI{20}{\nano\second} \\
Number of levels $N_\text{levels}$ & 3 (or 2 for validation) \\
GRAPE time slices $N$ & 50 \\
Gaussian $n_\sigma$ & 4 \\
\end{tabular}
\end{ruledtabular}
\end{table}

%==============================================================================
\section{Results}
\label{sec:results}
%==============================================================================

We present four experiments of increasing complexity: (A) two-level validation, (B) three-level gate-time sweep, (C) error budget decomposition, and (D) robustness analysis.

\subsection{Experiment A: Two-level validation}
\label{sec:2level}

As a validation step, we first compare pulse methods in the two-level (qubit-only) model where leakage is absent. Table~\ref{tab:2level} summarizes the closed-system fidelities for five single-qubit gates.

\begin{table}[!htbp]
\caption{Closed-system fidelity in the two-level model. GRAPE achieves unit fidelity for all gates. DRAG fidelity is low because the quadrature correction (designed to suppress $\ket{0}\!\leftrightarrow\!\ket{2}$ leakage) introduces a parasitic rotation in the two-level system where no $\ket{2}$ state exists. Gaussian and DRAG are defined only for X and Y gates.}
\label{tab:2level}
\begin{ruledtabular}
\begin{tabular}{lccc}
Gate & Gaussian & DRAG & GRAPE \\
\hline
X & 1.000000 & 0.747 & 1.000000 \\
Y & 1.000000 & 0.747 & 1.000000 \\
H & --- & --- & 1.000000 \\
S & --- & --- & 1.000000 \\
T & --- & --- & 1.000000 \\
\end{tabular}
\end{ruledtabular}
\end{table}

In the two-level model, the Gaussian pulse achieves near-unit fidelity because the $\pi$-rotation condition [Eq.~\eqref{eq:gaussian_amp}] is exact when no leakage channel exists. DRAG's low fidelity ($F = 0.747$) is expected: the quadrature correction $\Omega_Q = \beta\, d\Omega_I/dt$ is designed to cancel $\ket{1} \leftrightarrow \ket{2}$ transitions that do not exist in the two-level model, so it acts as an unwanted $\sigma_y$ rotation. GRAPE converges to unit fidelity for all five gates, with X and Y gates requiring only 7 iterations and T gate requiring 500.

This experiment confirms that the optimization framework is functioning correctly and that the comparison baseline (Gaussian) is properly calibrated.

\subsection{Experiment B: Three-level gate-time sweep}
\label{sec:3level}

The three-level model is where GRAPE's advantage becomes physical. Table~\ref{tab:3level} shows the X-gate fidelity and leakage probability $P_2$ as a function of gate time from 10 to \SI{100}{\nano\second}.

\begin{table}[!htbp]
\caption{Closed-system X-gate fidelity $F$ and leakage $P_2$ in the three-level transmon model as a function of gate time. GRAPE achieves $F = 1$ (to machine precision) with $P_2 < 10^{-15}$ at all gate times (GRAPE leakage column omitted). Gaussian fidelity improves monotonically with gate time as the Rabi frequency decreases. DRAG fidelity \emph{decreases} with gate time (see text).}
\label{tab:3level}
\begin{ruledtabular}
\begin{tabular}{l cc cc c}
& \multicolumn{2}{c}{Gaussian} & \multicolumn{2}{c}{DRAG} & {GRAPE} \\
$T$ (ns) & $F$ & $P_2$ & $F$ & $P_2$ & $F$ \\
\hline
10  & 0.786 & $10^{-1}$ & 0.935 & $5\!\times\!10^{-3}$ & 1.000 \\
15  & 0.939 & $10^{-2}$ & 0.900 & $6\!\times\!10^{-4}$ & 1.000 \\
20  & 0.972 & $6\!\times\!10^{-4}$ & 0.872 & $10^{-4}$ & 1.000 \\
30  & 0.988 & $3\!\times\!10^{-8}$ & 0.837 & $5\!\times\!10^{-8}$ & 1.000 \\
50  & 0.996 & $10^{-9}$ & 0.804 & $3\!\times\!10^{-9}$ & 1.000 \\
100 & 0.999 & $3\!\times\!10^{-10}$ & 0.777 & $10^{-9}$ & 1.000 \\
\end{tabular}
\end{ruledtabular}
\end{table}

Three qualitative features are evident:

\textit{GRAPE achieves unit fidelity at all gate times.} The numerical optimizer exploits both control channels to construct pulse shapes that simultaneously implement the target rotation in the computational subspace and suppress all leakage to $\ket{2}$. The leakage probabilities are at or below machine precision ($< 10^{-15}$) for gate times up to \SI{20}{\nano\second} and remain below $10^{-10}$ at longer gate times.

\textit{Gaussian fidelity improves monotonically.} As the gate time increases, the required Rabi frequency decreases [Eq.~\eqref{eq:gaussian_amp}], reducing the driving strength relative to the anharmonicity and suppressing leakage. At $T = \SI{100}{\nano\second}$, the Gaussian achieves $F = 0.999$ with negligible leakage.

\textit{DRAG fidelity decreases with gate time.} This counterintuitive result arises because the DRAG parameter $\beta = -\alpha/(2\Omega_\text{max})$ [Eq.~\eqref{eq:beta}] grows as the peak Rabi frequency $\Omega_\text{max}$ decreases with longer gate times. At \SI{100}{\nano\second}, $\beta \approx 12.5$ and the quadrature correction overcorrects, introducing a large parasitic rotation. This highlights a limitation of the analytical DRAG formula: it is derived perturbatively for $|\beta| \ll 1$ and loses accuracy when the correction becomes comparable to the primary pulse~\cite{gambetta2011analytic, theis2018counteracting}.

\subsection{Experiment C: Error budget analysis}
\label{sec:error_budget}

To decompose the error sources, we evaluate each pulse method under progressively more realistic noise models using the Lindblad master equation [Eq.~\eqref{eq:lindblad}] with IQM Garnet parameters ($T_1 = \SI{37}{\micro\second}$, $T_2 = \SI{9.6}{\micro\second}$) at a gate time of \SI{20}{\nano\second}. Table~\ref{tab:error_budget} presents the three-level error budget.

\begin{table}[!htbp]
\caption{Error budget for the X gate in the three-level transmon model at $T = \SI{20}{\nano\second}$. Infidelity $\epsilon = 1 - F$ and leakage $P_2$ are shown for each noise channel. ``Coherent'' denotes closed-system (unitary) evolution. Control noise entries show the mean infidelity over 100 random realizations with the indicated relative amplitude noise level.}
\label{tab:error_budget}
\begin{ruledtabular}
\begin{tabular}{l ccc}
Error source & Gaussian $\epsilon$ & DRAG $\epsilon$ & GRAPE $\epsilon$ \\
\hline
Coherent        & $2.8\!\times\!10^{-2}$ & $1.3\!\times\!10^{-1}$ & $<\!10^{-15}$ \\
$T_1$ only      & $2.8\!\times\!10^{-2}$ & $1.3\!\times\!10^{-1}$ & $2.1\!\times\!10^{-4}$ \\
$T_2$ only      & $2.8\!\times\!10^{-2}$ & $1.3\!\times\!10^{-1}$ & $5.8\!\times\!10^{-4}$ \\
$T_1 + T_2$     & $2.9\!\times\!10^{-2}$ & $1.3\!\times\!10^{-1}$ & $7.2\!\times\!10^{-4}$ \\
Noise 1\%       & $2.8\!\times\!10^{-2}$ & $1.3\!\times\!10^{-1}$ & $2.4\!\times\!10^{-4}$ \\
Noise 5\%       & $3.4\!\times\!10^{-2}$ & $1.4\!\times\!10^{-1}$ & $5.9\!\times\!10^{-3}$ \\
\end{tabular}
\end{ruledtabular}
\end{table}

The key finding is the separation of error regimes. For Gaussian and DRAG pulses, coherent error (leakage and rotation error) dominates: adding $T_1$ and $T_2$ decoherence changes the infidelity by less than $1\%$ relative. For GRAPE, the coherent error vanishes to machine precision, and the total error under decoherence ($7.2 \times 10^{-4}$) represents the \emph{decoherence floor}---the minimum achievable infidelity set by the hardware's $T_1$ and $T_2$ for this gate time.

The decoherence floor can be estimated analytically. For a gate of duration $T$ with relaxation rate $\gamma_1 = 1/T_1$ and dephasing rate $\gamma_\phi = 1/T_2 - 1/(2T_1)$, the leading-order infidelity contributions are~\cite{wood2018quantification}
\begin{equation}
\epsilon_{T_1} \approx \frac{T}{2T_1}, \quad \epsilon_\phi \approx \frac{T}{T_2} - \frac{T}{2T_1}.
\label{eq:decoherence_floor}
\end{equation}
For $T = \SI{20}{\nano\second}$, $T_1 = \SI{37}{\micro\second}$, $T_2 = \SI{9.6}{\micro\second}$, this gives $\epsilon_{T_1} \approx 2.7 \times 10^{-4}$ and $\epsilon_\phi \approx 1.8 \times 10^{-3}$, consistent with GRAPE's measured $T_1$-only infidelity of $2.1 \times 10^{-4}$ and $T_2$-only infidelity of $5.8 \times 10^{-4}$.

GRAPE's infidelity is $39\times$ smaller than Gaussian and $178\times$ smaller than DRAG in the three-level model with full decoherence.

Table~\ref{tab:error_budget_2level} shows the corresponding two-level error budget for comparison. In the two-level model (no leakage), Gaussian and GRAPE perform identically---both are decoherence-limited. This confirms that GRAPE's advantage in the three-level model arises entirely from leakage suppression, not from a superior two-level rotation.

\begin{table}[!htbp]
\caption{Error budget for the X gate in the two-level model at $T = \SI{20}{\nano\second}$. Without leakage, Gaussian and GRAPE achieve identical decoherence-limited performance.}
\label{tab:error_budget_2level}
\begin{ruledtabular}
\begin{tabular}{l ccc}
Error source & Gaussian $\epsilon$ & DRAG $\epsilon$ & GRAPE $\epsilon$ \\
\hline
Coherent        & $<\!10^{-8}$           & $2.5\!\times\!10^{-1}$ & 0 \\
$T_1$ only      & $2.5\!\times\!10^{-4}$ & $2.5\!\times\!10^{-1}$ & $2.4\!\times\!10^{-4}$ \\
$T_2$ only      & $3.8\!\times\!10^{-4}$ & $2.5\!\times\!10^{-1}$ & $3.8\!\times\!10^{-4}$ \\
$T_1 + T_2$     & $5.8\!\times\!10^{-4}$ & $2.5\!\times\!10^{-1}$ & $5.7\!\times\!10^{-4}$ \\
Noise 1\%       & $2.3\!\times\!10^{-4}$ & $2.5\!\times\!10^{-1}$ & $2.3\!\times\!10^{-4}$ \\
Noise 5\%       & $5.7\!\times\!10^{-3}$ & $2.6\!\times\!10^{-1}$ & $5.7\!\times\!10^{-3}$ \\
\end{tabular}
\end{ruledtabular}
\end{table}

\subsection{Experiment D: Robustness analysis}
\label{sec:robustness}

We evaluate robustness by sweeping two common calibration errors: (i) qubit frequency detuning $\delta\omega/2\pi \in [-5, +5]$ MHz, implemented as an additional term $\delta\omega \cdot \hat{n}$ in the Hamiltonian, and (ii) systematic amplitude error $\epsilon_a \in [-5\%, +5\%]$, implemented by scaling all pulse amplitudes by $(1 + \epsilon_a)$. Each sweep uses 41 linearly spaced points.

Table~\ref{tab:robustness} summarizes the results for the three-level model.

\begin{table}[!htbp]
\caption{Robustness of X-gate fidelity in the three-level model under detuning ($\delta\omega/2\pi \in \pm\SI{5}{\mega\hertz}$) and amplitude error ($\epsilon_a \in \pm 5\%$). ``Nom.'' is the nominal (zero-error) fidelity; ``Min'' and ``Mean'' are over the sweep range.}
\label{tab:robustness}
\begin{ruledtabular}
\begin{tabular}{l ccc ccc}
& \multicolumn{3}{c}{Detuning} & \multicolumn{3}{c}{Amplitude} \\
Method & Nom. & Min & Mean & Nom. & Min & Mean \\
\hline
Gaussian & 0.972 & 0.937 & 0.969 & 0.972 & 0.965 & 0.970 \\
DRAG     & 0.873 & 0.810 & 0.871 & 0.873 & 0.855 & 0.871 \\
GRAPE    & 1.000 & 0.931 & 0.976 & 1.000 & 0.994 & 0.998 \\
\end{tabular}
\end{ruledtabular}
\end{table}

GRAPE shows excellent amplitude robustness: even at $\pm 5\%$ amplitude error, the minimum fidelity remains 0.994, significantly above both Gaussian (0.965) and DRAG (0.855). This suggests that the GRAPE-optimized pulse shape naturally provides some resilience to amplitude miscalibration.

For detuning, GRAPE's minimum fidelity (0.931) is slightly lower than Gaussian's (0.937), though GRAPE's mean fidelity (0.976) exceeds Gaussian's (0.969). The modestly increased detuning sensitivity is expected: GRAPE pulses are spectrally more complex and can have narrower spectral features that are more sensitive to frequency offsets~\cite{propson2022robust}. This motivates incorporating detuning robustness directly into the GRAPE cost function, as demonstrated in robust optimal control frameworks~\cite{wilhelm2020introduction, goerz2014optimal}.

%==============================================================================
\section{Discussion}
\label{sec:discussion}
%==============================================================================

\subsection{The three-level model as the appropriate comparison arena}

Our results clearly demonstrate that the two-level model is inadequate for comparing GRAPE to analytical pulse methods. In two levels, a properly calibrated Gaussian pulse achieves near-unit fidelity (Table~\ref{tab:2level}), and there is nothing for GRAPE to improve upon---both methods hit the decoherence floor. The three-level model introduces leakage as a physically relevant error source that analytical corrections (DRAG) only partially address and that GRAPE can fully suppress.

This finding has implications for how GRAPE results are reported in the literature. Claims of large fidelity improvements based on two-level comparisons against uncalibrated baselines may overstate GRAPE's practical advantage. The meaningful comparison is in the multi-level model with a properly calibrated analytical baseline.

\subsection{The DRAG overcorrection anomaly}

The monotonic decrease of DRAG fidelity with gate time (Table~\ref{tab:3level}) deserves further discussion. The DRAG parameter $\beta = -\alpha/(2\Omega_\text{max})$ was derived under the assumption that the correction is perturbatively small~\cite{motzoi2009simple}. As the gate time increases and $\Omega_\text{max}$ decreases, $\beta$ grows and the perturbative approximation breaks down. At $T = \SI{100}{\nano\second}$, the quadrature amplitude $\Omega_Q \sim \beta\, d\Omega_I/dt$ can become comparable to $\Omega_I$ itself, violating the assumptions of the derivation.

In experimental practice, $\beta$ is typically calibrated empirically rather than computed from the analytical formula, and DRAG achieves high fidelities~\cite{chen2016measuring, sheldon2016procedure}. Our results highlight that the \emph{analytical} DRAG formula has a limited regime of validity and that numerical calibration or higher-order corrections~\cite{theis2018counteracting, motzoi2013backaction} are essential for accurate DRAG implementation.

\subsection{Error hierarchy and the decoherence floor}

The error budget (Table~\ref{tab:error_budget}) reveals a clear hierarchy for the three-level model at \SI{20}{\nano\second}:

\begin{enumerate}
\item \emph{Coherent error} (leakage + rotation error): Dominates for Gaussian ($\epsilon = 2.8 \times 10^{-2}$) and DRAG ($\epsilon = 1.3 \times 10^{-1}$). Eliminated by GRAPE ($\epsilon < 10^{-15}$).

\item \emph{Dephasing} ($T_2$): For GRAPE, contributes $5.8 \times 10^{-4}$ to infidelity. This is the dominant decoherence channel given $T_2 = \SI{9.6}{\micro\second} \ll 2T_1 = \SI{74}{\micro\second}$.

\item \emph{Relaxation} ($T_1$): Contributes $2.1 \times 10^{-4}$, smaller than dephasing due to the longer $T_1$.

\item \emph{Control noise} ($\sigma = 1\%$): Contributes $2.4 \times 10^{-4}$, comparable to $T_1$. At $\sigma = 5\%$, control noise ($5.9 \times 10^{-3}$) exceeds the decoherence floor and becomes the dominant error source.
\end{enumerate}

This hierarchy suggests that for IQM Garnet--class hardware, improving $T_2$ (through better materials, filtering, or dynamical decoupling) would have the greatest impact on GRAPE-optimized gate fidelity, while control noise at the $\sim 1\%$ level is already subdominant.

\subsection{Limitations}

Several limitations of this work should be acknowledged:

\begin{itemize}
\item \textit{Simulation only.} All results are from numerical simulation. No pulses were executed on physical hardware. The framework demonstrates API connectivity to IQM Garnet and retrieves system topology, but hardware validation remains future work.

\item \textit{Closed-system GRAPE.} The GRAPE optimization uses closed-system (unitary) fidelity as the cost function. Open-system GRAPE~\cite{schulte2011optimal} incorporating Lindblad dynamics during optimization could potentially discover pulses that partially compensate for decoherence, further improving performance.

\item \textit{Markovian noise.} The Lindblad model assumes Markovian dynamics. Low-frequency $1/f$ noise and non-Markovian effects~\cite{norris2016qubit}, which are significant in superconducting qubits, are not captured.

\item \textit{Single-qubit gates only.} Extension to two-qubit gates (CZ, CNOT) is computationally more demanding and involves additional physics (e.g., parasitic ZZ coupling).

\item \textit{Static parameters.} We use fixed calibration parameters; real hardware exhibits temporal drift in $\omega_q$, $T_1$, and $T_2$.

\item \textit{DRAG calibration.} We use the analytical $\beta$ formula rather than numerical optimization of $\beta$. Numerically optimized DRAG would likely outperform our analytical DRAG results, though it would still be limited to first-order leakage suppression.
\end{itemize}

%==============================================================================
\section{Conclusion}
\label{sec:conclusion}
%==============================================================================

We have presented a systematic comparison of Gaussian, DRAG, and GRAPE control pulses for single-qubit gates in a three-level transmon model with hardware-representative parameters. The principal findings are:

(1) GRAPE achieves unit closed-system fidelity ($1 - F < 10^{-15}$) with negligible leakage ($P_2 < 10^{-15}$) across all tested gate times (10--\SI{100}{\nano\second}), while Gaussian and DRAG pulses retain significant coherent errors.

(2) Under realistic decoherence ($T_1 = \SI{37}{\micro\second}$, $T_2 = \SI{9.6}{\micro\second}$), GRAPE reaches the decoherence floor ($\epsilon = 7.2 \times 10^{-4}$), outperforming Gaussian by $39\times$ and DRAG by $178\times$ in the three-level model.

(3) The analytical DRAG parameter $\beta = -\alpha/(2\Omega_\text{max})$ leads to overcorrection at long gate times, causing DRAG fidelity to \emph{decrease} monotonically---a result that highlights the limited regime of validity of first-order perturbative corrections.

(4) GRAPE pulses exhibit excellent amplitude robustness (fidelity $> 0.994$ over $\pm 5\%$ error) but modest detuning sensitivity compared to Gaussian pulses, motivating future robust optimization approaches.

These results quantify the regime where numerical optimal control provides a genuine advantage over analytical pulse methods: the multi-level transmon model where leakage is the dominant error source. Future work will extend to open-system GRAPE optimization, two-qubit entangling gates, robust cost functions incorporating parameter uncertainty~\cite{propson2022robust, goerz2014optimal}, and hardware validation on physical transmon processors.

\begin{acknowledgments}
The author thanks Embry-Riddle Aeronautical University for research support. The QubitPulseOpt framework is open source and available at \url{https://github.com/rylanmalarchick/QubitPulseOpt}.
\end{acknowledgments}

\appendix

\section{AI Disclosure}
AI-assisted tools (Claude, Anthropic) were used for code development, debugging, and manuscript preparation. All scientific concepts, experimental design, analysis, and interpretation were performed by the author. The author takes full intellectual responsibility for all content.

\bibliography{references}

\end{document}
